\documentclass[12pt,
     DIV=12,
     twoside=false,
     twocolumn=false,
     abstraction,
     %headsepline,
     %footsepline,
     %plainheadsepline,
     %plainfootsepline,
     dottedtoc,
     headings=normal,
     headinclude=true,
     footinclude=true,
     parskip=half]{scrbook}

\usepackage{subfiles}

\usepackage[USenglish]{babel}
% for German text, change this to
% \usepackage[ngerman]{babel}
\usepackage[utf8]{inputenc}
\usepackage[T1]{fontenc}

% avoid old unsafe commands
\RequirePackage[l2tabu,orthodox,abort]{nag}
\usepackage{fixltx2e}

\usepackage{typearea}
% \usepackage{layouts}

\usepackage[onehalfspacing]{setspace}
\raggedbottom
\usepackage{graphicx}

% font tweaks
\usepackage{ellipsis,ragged2e,marginnote}
\usepackage[tracking=true]{microtype}
\usepackage{cmbright}
%\usepackage[upright]{fourier}
\usepackage{inconsolata} % better monospaced font!
\renewcommand{\familydefault}{\sfdefault}
\setkomafont{sectioning}{\normalcolor\bfseries}
\setkomafont{disposition}{\normalcolor\bfseries}

\usepackage[small, bf, up, hang]{caption}
\usepackage[percent]{overpic}

\usepackage{mathtools}
\usepackage{nicefrac}
% unit[2.3]{N} and unitfrac[4.3]{Nm}{s}
\usepackage{units}

\usepackage{booktabs}
\usepackage{tabularx}

%%% Bibliography management
% quite popular is natbib, read about it here
% https://en.wikibooks.org/wiki/LaTeX/Bibliography_Management
% \usepackage{natbib}
% for this document we use biblatex with the biber backend
% this works very well, also for special characters etc.
\usepackage[backend=bibtex,style=alphabetic]{biblatex}
% this file contains the literature database
\addbibresource{bibliography.bib}

\usepackage{multirow}
\usepackage{rotating}
\usepackage{sagetex}
\usepackage{color}

% usually, inside a text a LaTeX command needs a trailing \ or {}
% for a trailing space. xspace solves this by looking ahead and inserting
% it for you.
\usepackage{xspace}
% Hence, after this redefinition "\LaTeX is" properly gives "LaTeX is".
\let\orgLaTeX\LaTeX
\renewcommand*{\LaTeX}{\orgLaTeX\xspace}

% quick setup for page numbers in the footer
\usepackage[automark]{scrlayer-scrpage}
\clearpairofpagestyles
% \ihead{\leftmark}
% \ohead{\includegraphics[height=1cm]{IMG/Logo.JPG}}
\ofoot{\thepage}

\title{Multiple \LaTeX Source Files}
\subtitle{Howto handle this in SMC}
\author{Harald Schilly}

\begin{document}

\maketitle

\chapter{Introduction}

This is an overview about how to write a large \LaTeX document
with multiple input sources, where:

\begin{itemize}
\item all sub-documents can be previewed by rendering only the partial document,
\item the forward/inverse search works,
\item SageTeX~\cite{sagetex} is supported,
\item formulas, graphics and the advanced KOMAscript package are enabled.
\end{itemize}


\tableofcontents

\chapter{Setup}

The general setup of this document is best explained by reading the header
of the ``\texttt{master.tex}'' file.

\section{Bibliography}

The bibliography uses \texttt{biblatex} and \texttt{biber},
which works much more robust than the older \texttt{bibtex}.
The file ``\texttt{bibliography.bib}'' defines the references
in the usual bibtex format,
and the references work across all sub-documents.
It also directly supports special characters. For example, the \cite{komascript} reference contains the letter ``ü'',
which is processed and rendered without any issues.

The next chapters are sub-documents.
They briefly describe each individual aspect in more detail.

\subfile{10-make.tex}
\subfile{20-subfiles.tex}
\subfile{30-komascript.tex}
\subfile{40-sagetex.tex}

\chapter{Conclusion}

\printbibliography

\end{document}

%sagemathcloud={"latex_command":"make 'master.pdf'"}
